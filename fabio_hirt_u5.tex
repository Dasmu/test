\documentclass[11pt]{article}
\usepackage[ngerman]{babel}
\usepackage[utf8]{inputenc}
\date{\today}
\author{Fabio Hirt}
\title{CS102 \LaTeX -Übung}
\begin{document}
\maketitle

\section{Erster Abschnitt}
Lorem ipsum dolor sit amet, consetetur sadipscing elitr, sed diam nonumy eirmod tempor invidunt ut labore et dolore magna aliquyam erat, sed diam voluptua. At vero eos et accusam et justo duo dolores et ea rebum. Stet clita kasd gubergren, no sea takimata sanctus est Lorem ipsum dolor sit amet. Lorem ipsum dolor sit amet, consetetur sadipscing elitr, sed diam nonumy eirmod tempor invidunt ut labore et dolore magna aliquyam erat, sed diam voluptua. At vero eos et accusam et justo duo dolores et ea rebum. Stet clita kasd gubergren, no sea takimata sanctus est Lorem ipsum dolor sit amet.

Lorem ipsum dolor sit amet, consetetur sadipscing elitr, sed diam nonumy eirmod tempor invidunt ut labore et dolore magna aliquyam erat, sed diam voluptua. At vero eos et accusam et justo duo dolores et ea rebum. Stet clita kasd gubergren, no sea takimata sanctus est Lorem ipsum dolor sit amet. Lorem ipsum dolor sit amet, consetetur sadipscing elitr, sed diam nonumy eirmod tempor invidunt ut labore et dolore magna aliquyam erat, sed diam voluptua. At vero eos et accusam et justo duo dolores et ea rebum. Stet clita kasd gubergren, no sea takimata sanctus est Lorem ipsum dolor sit amet.

Lorem ipsum dolor sit amet, consetetur sadipscing elitr, sed diam nonumy eirmod tempor invidunt ut labore et dolore magna aliquyam erat, sed diam voluptua. At vero eos et accusam et justo duo dolores et ea rebum. Stet clita kasd gubergren, no sea takimata sanctus est Lorem ipsum dolor sit amet. Lorem ipsum dolor sit amet, consetetur sadipscing elitr, sed diam nonumy eirmod tempor invidunt ut labore et dolore magna aliquyam erat, sed diam voluptua. At vero eos et accusam et justo duo dolores et ea rebum. Stet clita kasd gubergren, no sea takimata sanctus est Lorem ipsum dolor sit amet.

\section{Ich war hier}
Mojentale Carlos

Grüss

\pagebreak
\section{Tabelle}
Lorem ipsum dolor sit amet, consetetur sadipscing elitr, sed diam nonumy eirmod tempor invidunt ut labore et dolore magna aliquyam erat, sed diam voluptua. At vero eos et accusam et justo duo dolores et ea rebum. Stet clita kasd gubergren, no sea takimata sanctus est Lorem ipsum dolor sit amet. Lorem ipsum dolor sit amet, consetetur sadipscing elitr, sed diam nonumy eirmod tempor invidunt ut labore et dolore magna aliquyam erat, sed diam voluptua. At vero eos et accusam et justo duo dolores et ea rebum. Stet clita kasd gubergren, no sea takimata sanctus est Lorem ipsum dolor sit amet.\\

\noindent Erreichte Punktzahlen Werkzeuge der Informatik:

\begin{table}[!th]

\begin{tabular}{r|c|c|r}

& \textbf{erreichte Punkte} & \textbf{mögliche Punkte} & \textbf{in \%} \\
\hline
1 & 10 & 10 & 100\% \\
2 & 10 & 10 & 100\% \\
3 & 10 & 10 & 100\% \\
4 & 10 & 10 & 100\% \\
\hline
& \textbf{40} & \textbf{40} & \textbf{100\%} \\

\end{tabular}
\caption{bisher erreichte Punktzahl}
\end{table}


\section{Formeln}

\subsection{Pythagoras}
Der Satz des Pythagoras errechnet sich wie folgt: $a^{2} + b^{2} = c^{2}$ .
Daraus können wir die Länge der Hypothenuse wie folgt berechnen: $c = \sqrt[]{a^{2} + b^{2}}$ . 

\subsection{Summe}
Wir können auch die Formel für eine Summe angeben:
\begin{equation} 
s = \sum_{i=1}^n i = \frac{n * (n + 1)}{2}
\end{equation}


\end{document}
